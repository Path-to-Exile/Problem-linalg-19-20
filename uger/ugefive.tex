% !TEX root = ‎⁨‎⁨/../../master.tex

\chapter{Uge 5}

	\section{Basis opgaver}

		\subsection{i}

			$v_1+v_2=(7,3)$. Med hensyn til standardbasen bliver koordinaterne $(7,3)$. Med hensyn til basen $(v_1,v_2)$ bliver det $(1,1)$.

		\subsection{ii}

			Da vi har basen $(v_1,v_2)$ for både domænet og codomænet bliver matricen bare
				\begin{align*}
					A=\left(\begin{array}{rr} {1} & {0} \\ {0} & {3} \end{array}\right).
				\end{align*}

	\section{Standard opgaver}

		\subsection{4.20}

			\paragraph{a} Indses f.eks. vha. \cite[Korollar 4.3.12]{hesselholt2017}, da rangen af matricen basen udspænder er $2$.

			\paragraph{b} Denne bliver 
				\begin{align*}
					P=\left(\begin{array}{rr} {2} & {3} \\ {5} & {7} \end{array}\right).
				\end{align*}

			\paragraph{c} Denne bliver den inverse af $P$ (se f.eks. \cite[Eks. 4.4.13]{hesselholt2017}), dvs.
				\begin{align*}
					P^{-1}=\left(\begin{array}{rr} {-7} & {5} \\ {3} & {-2} \end{array}\right).
				\end{align*}

			\paragraph{d} Per \cite[Eks. 4.4.18]{hesselholt2017} fås koordinaterne 
				\begin{align*}
					y = P^{-1}x = \left(\begin{array}{rr} {-7} & {5} \\ {3} & {-2} \end{array}\right) \begin{pmatrix} x_1 \\ x_2 \end{pmatrix} = \begin{pmatrix} -7x_1 + 5x_2 \\ 3x_1 - 2x_2 \end{pmatrix}.
				\end{align*}

		\subsection{4.21}

			\paragraph{a} Indses f.eks. vha. \cite[Korollar 4.3.12]{hesselholt2017}, da rangen af matricen basen udspænder er $3$.

			\paragraph{b} Denne bliver 
				\begin{align*}
					P=\left(\begin{array}{rrr} {1} & {-1} & {1} \\ {0} & {1} & {-1} \\ {0} & {0} & {1} \end{array}\right).
				\end{align*}

			\paragraph{c} Denne bliver den inverse af $P$ (se f.eks. \cite[Eks. 4.4.13]{hesselholt2017}), dvs.
				\begin{align*}
					P^{-1}=\left(\begin{array}{rrr} {1} & {1} & {0} \\ {0} & {1} & {1} \\ {0} & {0} & {1} \end{array}\right).
				\end{align*}

			\paragraph{d} Per \cite[Eks. 4.4.18]{hesselholt2017} får vi vektoren til 
				\begin{align*}
					y = \left(\begin{array}{rrr} {1} & {1} & {0} \\ {0} & {1} & {1} \\ {0} & {0} & {1} \end{array}\right) \begin{pmatrix} x_1 \\ x_2 \\ x_3 \end{pmatrix} = \begin{pmatrix} x_1 + x_2 \\ x_2 + x_3 \\ x_3 \end{pmatrix}.
				\end{align*}

		\subsection{4.22}

			\paragraph{a} Dette bliver matricen dannet af vektorene $(u_1,u_2,u_3)$, dvs.
				\begin{align*}
					P=\left(\begin{array}{rrr} 1 & 1 & 0 \\ 0 & 2 & 1 \\ 1 & 2 & 1 \end{array}\right).
				\end{align*}

			\paragraph{b} Dette bliver matricen dannet af vektorene $(v_1,v_2)$, dvs.
				\begin{align*}
					P=\left(\begin{array}{rr} 1 & 2 \\ 2 & 3 \end{array}\right).
				\end{align*}

			\paragraph{c} Per \cite[Sætning 4.4.14]{hesselholt2017} er dette $B=Q^{-1}AP$

			\paragraph{d} Dette bliver
				\begin{align*}
					B = Q^{-1}AP = \left(\begin{array}{rrr} {-1} & {16} & {10} \\ {3} & {3} & {0} \end{array}\right).
				\end{align*} 

			\paragraph{e} Tegn.

		\subsection{4.23}

			\paragraph{a} Dette bliver matricen dannet af vektorene $(v_1,v_2,v_3)$, dvs.
			    \begin{align*}
					P=\left(\begin{array}{rrr} 0 & 1 & 1 \\ 1 & 1 & 1 \\ 1 & 2 & 3 \end{array}\right).
				\end{align*}

			\paragraph{b} Per \cite[Sætning 4.4.14]{hesselholt2017} er dette $B=P^{-1}AP$

			\paragraph{c} Dette bliver
				\begin{align*}
					B = P^{-1}AP = \left(\begin{array}{rrr} {3} & {2} & {-1} \\ {0} & {-1} & {1} \\ 0 & 0 & 1 \end{array}\right).
				\end{align*} 

			\paragraph{d} Tegn.

		\subsection{4.24}

			\paragraph{a} Det har rang $3$

			\paragraph{b} Dette bliver matricen dannet af vektorene $(v_1,v_2,v_3)$, dvs.
			    \begin{align*}
					P=\left(\begin{array}{rrr} -1 & 1 & 0 \\ 1 & 0 & 1 \\ 1 & -1 & 1 \end{array}\right).
				\end{align*}

			\paragraph{c} Indses f.eks. fra en tegning at $A=PBP^{-1}$

			\paragraph{d} Vi får
				\begin{align*}
					A = PBP^{-1} = \left(\begin{array}{rrr} {-1} & {1} & {-2} \\ {2} & {2} & {0} \\ 3 & 0 & 2 \end{array}\right).
				\end{align*}
				
			\paragraph{e} Tegn.

		\subsection{4.25}

			\paragraph{a} Dette bliver matricen dannet af vektorene $(v_1, v_2, v_3)$, dvs.
			    \begin{align*}
					P = \left(\begin{array}{rrr} 0 & 1 & 1 \\ 1 & 0 & 1 \\ 1 & 1 & 0 \end{array}\right).
				\end{align*}

			\paragraph{b} Per \cite[Sætning 4.4.14]{hesselholt2017} er dette $B=P^{-1}AP$

			\paragraph{c} Dette bliver
				\begin{align*}
					B = P^{-1}AP = \left(\begin{array}{rrr} {2} & {0} & {0} \\ {0} & {-6} & {0} \\ 0 & 0 & 4 \end{array}\right).
				\end{align*} 

			\paragraph{d} Ja, $f$ er en isomorfi, da $B$ er invertibel (kvadratisk med fuld rang).

			\paragraph{e} Tegn.



	\section{Opgaver til fordybelse}

		\subsection{4.16}

			\paragraph{a} Oplagt.

			\paragraph{b} Skriv det ud og brug at $g$ er lineær.

		\subsection{4.17}
		    Bemærk: Denne opgave påstår fejlagtigt, at $(V, +, \cdot)$ fra opgave 4.7 er et reelt vektorrum. Dette opnås dog først ved at specialisere til $\mathbb{F} = \mathbb{R}$.

			\paragraph{a} Følger af linearitet af matricer

			\paragraph{b} Den naturlige basis er familien $(1, i)$. Således har dette reelle vektorrum dimension $2$.

			\paragraph{c} Den naturlige basis er familien
			\[
			\left(
			\begin{pmatrix} 1 & 0 \\ 0 & 0 \end{pmatrix},
			\begin{pmatrix} 0 & 1 \\ 0 & 0 \end{pmatrix},
			\begin{pmatrix} 0 & 0 \\ 1 & 0 \end{pmatrix},
			\begin{pmatrix} 0 & 0 \\ 0 & 1 \end{pmatrix}
			\right).
			\]
			Således har dette reelle vektorrum dimension $4$.


