% !TEX root = ‎⁨‎⁨/../../master.tex

\chapter{Uge 5}

	\section{Basis opgaver}

		\subsection{i}

			$v_1+v_2=(7,3)$. Med hensyn til standardbasen bliver koordinaterne $(7,3)$. Med hensyn til basen $(v_1,v_2)$ bliver det $(1,1)$.

		\subsection{ii}

			Da vi har basen $(v_1,v_2)$ for både domænet og co-domænet bliver matricen bare
				\begin{align*}
					A=\left(\begin{array}{rr} {1} & {0} \\ {0} & {3} \end{array}\right).
				\end{align*}

	\section{Standard opgaver}

		\subsection{4.20}

			\paragraph{a} Indses f.eks. vha. \cite[Korollar 4.3.12]{hesselholt2017}, da rangen af matricen basen udspænder er $2$.

			\paragraph{b} Denne bliver 
				\begin{align*}
					P=\left(\begin{array}{rr} {2} & {5} \\ {3} & {7} \end{array}\right).
				\end{align*}

			\paragraph{c} Denne bliver den inverse af P (se f.eks. \cite[Eks. 4.4.13]{hesselholt2017})
				\begin{align*}
					P^{-1}=\left(\begin{array}{rr} {-7} & {5} \\ {3} & {-2} \end{array}\right).
				\end{align*}

			\paragraph{d} Per \cite[Eks. 4.4.18]{hesselholt2017} får vi vektoren til 
				\begin{align*}
					\left(\begin{array}{rr} {-7} & {5} \\ {3} & {-2} \end{array}\right) (y_1,y_2)^T=(-7y_1+5y_2,3y_1-2y_2)^T.
				\end{align*}

		\subsection{4.21}

			\paragraph{a} Indses f.eks. vha. \cite[Korollar 4.3.12]{hesselholt2017}, da rangen af matricen basen udspænder er $3$.

			\paragraph{b} Denne bliver 
				\begin{align*}
					P=\left(\begin{array}{rrr} {1} & {0} & {0} \\ {-1} & {1} & {0} \\ {1} & {-1} & {1} \end{array}\right).
				\end{align*}

			\paragraph{c} Denne bliver den inverse af P (se f.eks. \cite[Eks. 4.4.13]{hesselholt2017})
				\begin{align*}
					P^{-1}=\left(\begin{array}{rrr} {1} & {0} & {0} \\ {1} & {1} & {0} \\ {0} & {1} & {1} \end{array}\right).
				\end{align*}

			\paragraph{d} Per \cite[Eks. 4.4.18]{hesselholt2017} får vi vektoren til 
				\begin{align*}
					\left(\begin{array}{rrr} {1} & {0} & {0} \\ {1} & {1} & {0} \\ {0} & {1} & {1} \end{array}\right) (y_1,y_2,y_3)^T=(y_1,y_1+y_2,y_2+y_3)^T.
				\end{align*}

		\subsection{4.22}

			\paragraph{a} Dette bliver matricen dannet af vektorene $(u_1,u_2,u_3)$.

			\paragraph{b} Dette bliver matricen dannet af vektorene $(v_1,v_2)$.

			\paragraph{c} Per \cite[Sætning 4.4.14]{hesselholt2017} er dette $B=Q^{-1}AP$

			\paragraph{d} Dette bliver
				\begin{align*}
					\left(\begin{array}{rrr} {-1} & {16} & {10} \\ {3} & {3} & {0} \end{array}\right).
				\end{align*} 

			\paragraph{e} Tegn.

		\subsection{4.23}

			\paragraph{a} Dette bliver matricen dannet af vektorene $(v_1,v_2,v_3)$.

			\paragraph{b} Per \cite[Sætning 4.4.14]{hesselholt2017} er dette $B=P^{-1}AP$

			\paragraph{c} Dette bliver
				\begin{align*}
					\left(\begin{array}{rrr} {3} & {2} & {-1} \\ {3} & {2} & {4} \\ -3 & -3 & -2 \end{array}\right).
				\end{align*} 

			\paragraph{d} Tegn.

		\subsection{4.24}

			\paragraph{a} Det har rang $3$

			\paragraph{b} Matricen med basisvektorene som søjler, se f.eks. \cite[Eks. 4.4.6]{hesselholt2017}.

			\paragraph{c} Indses f.eks. fra en tegning at $A=PBP^{-1}$

			\paragraph{d} Vi får
				\begin{align*}
					\left(\begin{array}{rrr} {-1} & {1} & {-2} \\ {2} & {2} & {0} \\ 3 & 0 & 2 \end{array}\right).
				\end{align*}

		\subsection{4.25}

			\paragraph{a} Dette bliver matricen dannet af vektorene $(v_1,v_2,v_3)$.

			\paragraph{b} Per \cite[Sætning 4.4.14]{hesselholt2017} er dette $B=P^{-1}AP$

			\paragraph{c} Dette bliver
				\begin{align*}
					\left(\begin{array}{rrr} {2} & {0} & {0} \\ {0} & {-6} & {0} \\ 0 & 0 & 4 \end{array}\right).
				\end{align*} 

			\paragraph{d} Ja, det er en isomorfi da $BB^{-1}=B^{-1}B=id_3$.

			\paragraph{e} Tegn.



	\section{Opgaver til fordybelse}

		\subsection{4.16}

			\paragraph{a} Oplagt.

			\paragraph{b} Skriv det ud og brug at $g$ er lineær.

		\subsection{4.17}

			\paragraph{a} Følger af linearitet af matricer

			\paragraph{b} $(1,0)$, $(0,1)$.

			\paragraph{c} De to matricer 
				\begin{align*}
					\left(\begin{array}{rr} {1} & {0} \\ {0} & {1} \end{array}\right) \quad \text{and} \quad \left(\begin{array}{rr} {0} & {-1} \\ {1} & {0} \end{array}\right).
				\end{align*} 


