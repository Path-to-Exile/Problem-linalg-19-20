% !TEX root = ‎⁨‎⁨/../../master.tex

\chapter{Uge 6}

	\section{Basis opgaver}

		\subsection{i}

			De er ortogonale, da det indre produkt er 0. De er dog ikke ortonormale, da vektorerne ikke er enhedsvektorer.

		\subsection{ii}

			Det er $\sqrt{4^2+0^2+3^2}=\sqrt{25}=5$.

	\section{Standard opgaver}

		\subsection{6.1}

			\subsubsection{a}

				Vi tjekker \cite[Definition 6.1.1]{hesselholt2017}, oplagt

			\subsubsection{b}

				$|x|=\sqrt{3+4}=\sqrt{7}$. $|y|=\sqrt{3\cdot 16+ 4 \cdot 9}=\sqrt{84}=2\sqrt{21}$. $|z|=\sqrt{3\cdot \sqrt{3}\sqrt{3} + 4\cdot (-2i)(2i)}=\sqrt{9 + 8}=\sqrt{17}$.

			\subsubsection{c}

				Indre produktet er $0$, $x$ og $y$ er da ortogonale.

			\subsubsection{d}

				De er de ikke, da deres standard indre produkt er $-1$.

		\subsection{6.2}

			\subsubsection{a}

				Følger af linearitet af integralet.

			\subsubsection{b}

				Det integrerer til $0$ og er derfor ortogonale. $1$ er oplagt en enhedsvektor med hensyn til indre produktet. At $\sqrt{3}(2x-1)$ er følger af en let udregning. 

			\subsubsection{c}

				Normen er $\sqrt{\frac{1}{2n+1}}$.

		\subsection{6.3}

			Vi har at 
				\begin{align*}
					(x_1+\ldots+x_n)^2=\langle v,1\rangle^2\leq ||1||^2||v||^2=n(x_1^2+\ldots+x_n^2),
				\end{align*} 
			hvor vi brugte Cauchy-Schwarz i uligheden.

		\subsection{6.4}

			Vi følger \cite[Eksempel 6.1.6]{hesselholt2017}. Indreprodukt af $\langle x,y \rangle = 3$. Vi får da vinklen til $\cos \theta = 1/2$, $\theta = \pi/3$.

		\subsection{6.5}

			Gram-Schmidt.

		\subsection{6.6}

			\subsubsection{a}

				Det ses at de er ortogonale og fra definition af lineært uafhængighed \cite[Definition 4.3.4]{hesselholt2017} er de også det.

			\subsubsection{b}

				Tag f.eks. $w_3=(0,0,1)$. Da er det en basis for $\real^3$ per \cite[Lemma 4.3.9]{hesselholt2017}.

			\subsubsection{c}

				Du ender med standardbasen for $\real^3$.

			\subsubsection{d}

				Oplagt da det er enhedsmatricen.

			\subsubsection{e}

				Oplagt igen.

		\subsection{6.7}

			\subsubsection{a}

				Det en basis for $\real^3$ per \cite[Lemma 4.3.9]{hesselholt2017}, at de er ortogonale eftervises let.

			\subsubsection{b}

				Linearitet følger af at standard indreproduktet er en indreprodukt.

			\subsubsection{c}
			
			Denne er forkert :)

				Der regnes og man får da matricen for A til
					\begin{align*}
						\left(\begin{array}{rrrr} 1 & -1 & 1 & 1 \\ 2 & 2 & 0 & 0 \\ 0 & 0 & -1 & 2 \\ -1 & 1 & -1 & 1 \end{array}\right).
					\end{align*} 
				Og ved brug af \cite[Eksempel 4.4.16]{hesselholt2017} bliver B 
					\begin{align*}
						\left(\begin{array}{rrrr} 0 & 0 & 0 & 0 \\ 2 & 1/4 & -1/4 & 1/4 \\ 0 & 7/2 & 1/2 & -1/2 \\ 0 & 1/2 & 7/2 & 1/2 \end{array}\right).
					\end{align*} 

			\subsubsection{d}

				Gøres i Maple\ldots

			\subsubsection{e}

				$f$ har har rang $3$, $f^{\circ 2}$ har rang $2$, $f^{\circ 3}$ har rang $1$, $f^{\circ 4}$ har rang $0$.



	\section{Opgaver til fordybelses}

		\subsection{Opgave 1}

			\subsubsection{a}

				Første del indses let, evt. ved Maple. Dette giver ortogonalitet. At det basis følger af den per definition udspænder $\text{Sig}_3$ og den er lineært uafhængig per \cite[Definition 4.3.4]{hesselholt2017}.

			\subsubsection{b}

				Divider med $\pi$ og vektorerne er ortonormale per a.

			\subsubsection{c}

				Følger af \cite[Sætning 6.2.6]{hesselholt2017} og at basen er ortonormal divideret med $\pi$.


			\subsubsection{d}

				












	



