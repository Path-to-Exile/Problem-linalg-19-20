\documentclass[11pt]{book}

\usepackage[utf8]{inputenc} %font
\usepackage[danish]{babel}	 %Language
\usepackage[usenames]{color}
% MATHEMATICS
\usepackage{amsmath}
\usepackage{amssymb}
\usepackage{amsthm}
\usepackage{mathtools}
\usepackage{mathrsfs}
\allowdisplaybreaks
%______________________________

% Bibliography and appendices
\usepackage[nottoc]{tocbibind}
\bibliographystyle{apalike}

\usepackage{appendix}
\renewcommand\appendixtocname{Appendikser} %Adds appendikser instead of appendices 

\usepackage{pageslts} % Smart pages

\usepackage{etoolbox} % Fix clearpage in book environment 
\makeatletter
\patchcmd{\chapter}{\if@openright\cleardoublepage\else\clearpage\fi}{}{}{}
\makeatother


\usepackage{titlesec} 
\titleformat{\chapter}{\normalfont\LARGE\bfseries}{\thechapter.}{20pt}{\LARGE} %fixing chapter size
\titlespacing*{\chapter}{0pt}{20pt}{20pt} %fixing chapter spacing

%Fixing continuous indexing over chapter
\usepackage{chngcntr}
\counterwithout{footnote}{chapter}
%______________________________

% RANDOM
\usepackage[table,xcdraw,svgnames]{xcolor} % Additional colors
%\usepackage{wasysym} % Funny symbols
%\usepackage{listings} % Displaying code, e.g. Python in latex
% Defining R environment within listings
%\usepackage[lmargin=1.5cm,rmargin=2.5cm]{geometry}
\usepackage[margin=2.5cm, tmargin=2.5cm]{geometry} % Easy interface for page dimension. (laver margener rigtige)
\usepackage{enumitem} % Control to enumerate/itemize environment 
%\usepackage{lastpage} % Last page with ease
\usepackage{titling} %Titlepage without number
\usepackage[indentfirst=false,rightmargin=0pt]{quoting} %package to make qoutes, only used if first line should no be indented
\usepackage[colorinlistoftodos,prependcaption,textsize=tiny]{todonotes} %Todo notes
\linespread{1.3} % Line spread
%______________________________

% Graphics 
\usepackage{graphicx} %Inserting pictures
\usepackage{subcaption} %Inserting multiple pictures in one figure
% Quite big packages, uncomment if needed.
%\usepackage{tikz} 
%\usepackage{pgfplots}
%\usepackage{hyperrefpgfplots}
%\pgfplotsset{compat=1.13} 
%\usepgfplotslibrary{fillbetween}
%\usetikzlibrary{positioning}
%______________________________

% Header 
\usepackage{fancyhdr} 
\usepackage{textcase}
\pagestyle{fancyplain}
\setlength{\headheight}{16pt} %Fixing warning about headhight
%______________________________

\usepackage[colorlinks]{hyperref}
\AtBeginDocument{%
  \hypersetup{
    citecolor=SteelBlue,
    linkcolor=SteelBlue,   
    urlcolor=SteelBlue}}
%Colorlinks for colored citation insted of boxes. SVG color names in hypersetup
\DeclarePairedDelimiter{\ceil}{\lceil}{\rceil}

% Custom Commands
\newcommand{\setm}{\setminus}
\newcommand{\n}{\mathbb{N}}
\newcommand{\z}{\mathbb{Z}}
\newcommand{\q}{\mathbb{Q}}
\newcommand{\cx}{\mathbb{C}}
\newcommand{\real}{\mathbb{R}}
\newcommand{\field}{\mathbb{F}}
\newcommand{\rz}{\mathfrak{R}z}
\newcommand{\im}{\mathfrak{I}}
\newcommand{\oneton}{\{1,2,3,...,n\}}
\newcommand\myeq{\mathrel{\stackrel{\makebox[0pt]{\mbox{\normalfont\tiny def}}}{=}}}
\newcommand\myeqq{\mathrel{\stackrel{\makebox[0pt]{\mbox{\normalfont\tiny (D2)}}}{=}}}
\DeclareMathOperator{\interior}{int}

% Theorem etc. environments
\newtheorem{theorem}{Theorem}[section]
\newtheorem{proposition}{Proposition}[section]
\newtheorem{corollary}{Corollary}[theorem]
\newtheorem{lemma}{Lemma}[theorem]
\newtheorem{definition}{Definition}[section]
\newtheorem{formodning}{Formodning}[section]
\newtheorem{liste}{List}
%______________________________

\usepackage{todonotes}
\usepackage{comment} %use to comment large chunks of document out 

% Header and Foot.
\fancyhead[R]
	{\footnotesize{\textit{Anton Suhr}}
	}
\fancyhead[C]
	{\footnotesize{\textit{LinAlg 19/20}}
	}
\fancyhead[L]
	{\textit{\footnotesize\scshape\thepage}
	} 
\fancyfoot[L]{} % Empty left footer
\fancyfoot[C]{} % Empty center footer
\fancyfoot[R]{} % Page numbering for right footer
%(\theCurrentPage) af \lastpageref{LastPages}

%______________________________ MAKETTITLE
\title{Løsninger til opgaver på den korte dag i Linæer Algebra 19/20}
\author{Anton Suhr}
\date{\today}


%______________________________

\begin{document}


\frontmatter

		%\maketitle
		\tableofcontents
		
\mainmatter 

Alle tal, f.eks. 2.4, refererer til opgaver i \cite{hesselholt2017}. Opgaver med bogstaver refererer til ugesedler på Canvas. Det er yderligere indforstået hvorvidt en given variabel er en vektor eller skalar.

%	% !TEX root = ‎⁨‎⁨/../../master.tex

\chapter{Uge 1}

	\section{Basisopgaver}

		\subsection{i}

		Angiv totalmatricen for ligningssystemet
		\begin{align*}
			\left\{\begin{aligned} x_{1}+7 x_{2} &=-1 \\ 3 x_{1}+4 x_{2} &=-4 \end{aligned}\right.
		\end{align*} 
		Per \cite[Eksempel 1.1.2]{hesselholt2017} får vi at
			\begin{equation}
				\left(\begin{array}{rr|r} {1} & {7} & {-1} \\ {3} & {4} & {-4} \end{array}\right)
			\end{equation}
		er totalmatrixen for ligningssytemet.

		\subsection{ii}

		Er
			\begin{equation}
				\left(\begin{array}{lll}{1} & {0} & {0} \\ {1} & {0} & {0} \\ {0} & {0} & {1}\end{array}\right)
			\end{equation}
		på echelon form?

		Den opfylder ikke betingelse $(1)$ i \cite[Def. 1.2.7]{hesselholt2017}, da den har to ledende indgange over hinanden. Den er derfor ikke på echelon form. Flyene \textbf{skal} således flyve i vifte.

	\section{Standardopgaver}

		\subsection{1.1}

		Vi får matricen på reduceret echelonform, f.eks. ved:
			\begin{align*}
				A=&\left(\begin{array}{cccccc}{1} & {-2} & {3} & {2} & {1} & {10} \\ {2} & {-4} & {8} & {3} & {10} & {7} \\ {3} & {-6} & {10} & {6} & {5} & {27}\end{array}\right) \\
				&\left(\begin{array}{cccccc}{1} & {-2} & {3} & {2} & {1} & {10} \\ {2} & {-4} & {8} & {3} & {10} & {7} \\ {3} & {-6} & {10} & {6} & {5} & {27}\end{array}\right) \begin{array}{c} \, \\ -2R_1 \\ -3R_1 \end{array}\\
				&\left(\begin{array}{cccccc}{1} & {-2} & {3} & {2} & {1} & {10} \\ {0} & {0} & {2} & {-1} & {8} & {-13} \\ {0} & {0} & {1} & {0} & {2} & {-3}\end{array}\right) \begin{array}{c} -3R_3 \\ -2R_3 \\ \, \end{array}\\
				&\left(\begin{array}{cccccc}{1} & {-2} & {0} & {2} & {-5} & {19} \\ {0} & {0} & {0} & {-1} & {4} & {-7} \\ {0} & {0} & {1} & {0} & {2} & {-3}\end{array}\right) \begin{array}{c} -3R_3 \\ -2R_3 \\ \, \end{array}\\
				&\left(\begin{array}{cccccc}{1} & {-2} & {0} & {2} & {-5} & {19} \\ {0} & {0} & {0} & {-1} & {4} & {-7} \\ {0} & {0} & {1} & {0} & {2} & {-3}\end{array}\right) \begin{array}{c} 2R_2 \\ -1 \\ R_2\leftrightarrow R_3 \end{array}\\
				A'=&\left(\begin{array}{cccccc}{1} & {-2} & {0} & {0} & {3} & {5} \\ {0} & {0} & {1} & {0} & {2} & {-3} \\ {0} & {0} & {0} & {1} & {-4} & {7} \end{array}\right) 
			\end{align*} 

		\subsection{1.2}

		Vi får matricen på reduceret echelonform, f.eks. ved:
			\begin{align*}
				B=&\left(\begin{array}{llll}{1} & {2} & {1} & {4} \\ {3} & {8} & {7} & {20} \\ {2} & {7} & {9} & {23}\end{array}\right) \\
				&\left(\begin{array}{llll}{1} & {2} & {1} & {4} \\ {3} & {8} & {7} & {20} \\ {2} & {7} & {9} & {23}\end{array}\right) \begin{array}{c} \, \\ -3R_1 \\ -2R_1 \end{array}\\
				&\left(\begin{array}{llll}{1} & {2} & {1} & {4} \\ {0} & {2} & {4} & {8} \\ {0} & {3} & {7} & {15}\end{array}\right) \begin{array}{c} -R_2 \\ 1/2 \\ -3/2R_2 \end{array}\\
				&\left(\begin{array}{llll}{1} & {0} & {-3} & {-4} \\ {0} & {1} & {2} & {4} \\ {0} & {0} & {1} & {3}\end{array}\right) \begin{array}{c} -R_2 \\ 1/2 \\ -3/2R_2 \end{array}\\
				&\left(\begin{array}{llll}{1} & {0} & {-3} & {-4} \\ {0} & {1} & {2} & {4} \\ {0} & {0} & {1} & {3}\end{array}\right) \begin{array}{c} 3R_3 \\ -2R_3 \\ \, \end{array}\\
				B'=&\left(\begin{array}{llll}{1} & {0} & {0} & {5} \\ {0} & {1} & {0} & {-2} \\ {0} & {0} & {1} & {3}\end{array}\right) 
			\end{align*} 

		\subsection{1.4}

		Vi bruger her \cite[Sætning 1.2.18]{hesselholt2017}

		\subsubsection{a}

		Vi har fået givet 
			\begin{align*}
				\text { a) } \quad\left\{\begin{array}{r}{2 x_{1}-x_{2}+x_{3}=3} \\ {-x_{1}+2 x_{2}+4 x_{3}=6} \\ {x_{1}+x_{2}+5 x_{3}=9}\end{array}\right.
			\end{align*}
		Den tilhørende totalmatrice er da
			\begin{equation}
				A=\left(\begin{array}{rrr|r} {2} & {-1} & {1} & {3} \\ {-1} & {2} & {4} & {6} \\ {1} & {1} & {5} & {9} \end{array}\right).
			\end{equation}
		Denne løses:
			\begin{align*}
				&\left(\begin{array}{rrr|r} {2} & {-1} & {1} & {3} \\ {-1} & {2} & {4} & {6} \\ {1} & {1} & {5} & {9} \end{array}\right) \begin{array}{c} -2R_3 \\ R_3 \\ \, \end{array}\\
				&\left(\begin{array}{rrr|r} {0} & {-3} & {-9} & {-15} \\ {0} & {3} & {9} & {15} \\ {1} & {1} & {5} & {9} \end{array}\right) \begin{array}{c} R_2 \\ 1/3 \\ -1/3R_2 \end{array}\\
				&\left(\begin{array}{rrr|r} {0} & {0} & {0} & {0} \\ {0} & {1} & {3} & {5} \\ {1} & {0} & {2} & {4} \end{array}\right)\\
				A'=&\left(\begin{array}{rrr|r}  {1} & {0} & {2} & {4} \\ {0} & {1} & {3} & {5} \\ {0} & {0} & {0} & {0} \end{array}\right).
			\end{align*}
		Vi har at $r=2<3=n$ vi er derfor i tilfælde $(4)$. Vi aflæser løsningsmængden til
			\begin{align*}
				x =\left(\begin{array}{c} 4-2t \\ 5-3t \\ t \end{array}\right).
			\end{align*} 
		\subsubsection{b}

		Vi har fået givet 
			\begin{align*}
				\text { b) } \quad\left\{\begin{array}{r}{2 x_{1}-x_{2}+x_{3}=4} \\ {-x_{1}+2 x_{2}+4 x_{3}=6} \\ {x_{1}+x_{2}+5 x_{3}=9}\end{array}\right.
			\end{align*}
		Den tilhørende totalmatrice er da
			\begin{equation}
				B=\left(\begin{array}{rrr|r} {2} & {-1} & {1} & {4} \\ {-1} & {2} & {4} & {6} \\ {1} & {1} & {5} & {9} \end{array}\right).
			\end{equation}
		Denne løses:
			\begin{align*}
				&\left(\begin{array}{rrr|r} {2} & {-1} & {1} & {4} \\ {-1} & {2} & {4} & {6} \\ {1} & {1} & {5} & {9} \end{array}\right) \begin{array}{c} -2R_3 \\ R_3 \\ \, \end{array}\\
				&\left(\begin{array}{rrr|r} {0} & {-3} & {-9} & {-14} \\ {0} & {3} & {9} & {15} \\ {1} & {1} & {5} & {9} \end{array}\right) \begin{array}{c} R_2 \\ 1/3 \\ -1/3R_2 \end{array}\\
				&\left(\begin{array}{rrr|r} {0} & {0} & {0} & {1} \\ {0} & {1} & {3} & {5} \\ {1} & {0} & {2} & {4} \end{array}\right)\\
				B'=&\left(\begin{array}{rrr|r}  {1} & {0} & {2} & {4} \\ {0} & {1} & {3} & {5} \\ {0} & {0} & {0} & {1} \end{array}\right).
			\end{align*}
		Vi er nu i tilfælde $(2)$, ligningssystemet har da ingen løsninger.

		\subsubsection{c}

		Vi har fået givet 
			\begin{align*}
				\text { c) } \quad\left\{\begin{array}{r}{2 x_{1}-x_{2}+2x_{3}=4} \\ {-x_{1}+2 x_{2}+4 x_{3}=6} \\ {x_{1}+x_{2}+5 x_{3}=9}\end{array}\right.
			\end{align*}
		Den tilhørende totalmatrice er da
			\begin{equation}
				C=\left(\begin{array}{rrr|r} {2} & {-1} & {2} & {4} \\ {-1} & {2} & {4} & {6} \\ {1} & {1} & {5} & {9} \end{array}\right).
			\end{equation}
		Denne løses:
			\begin{align*}
				&\left(\begin{array}{rrr|r} {2} & {-1} & {2} & {4} \\ {-1} & {2} & {4} & {6} \\ {1} & {1} & {5} & {9} \end{array}\right) \begin{array}{c} -2R_3 \\ R_3 \\ \, \end{array}\\
				&\left(\begin{array}{rrr|r} {0} & {-3} & {-8} & {-14} \\ {0} & {3} & {9} & {15} \\ {1} & {1} & {5} & {9} \end{array}\right) \begin{array}{c} R_2 \\ 1/3 \\ -1/3R_2 \end{array}\\
				&\left(\begin{array}{rrr|r} {0} & {0} & {1} & {1} \\ {0} & {1} & {3} & {5} \\ {1} & {0} & {2} & {4} \end{array}\right) \begin{array}{c} \, \\ -3R_1 \\ -2R_1 \end{array}\\
				&\left(\begin{array}{rrr|r} {0} & {0} & {1} & {1} \\ {0} & {1} & {0} & {2} \\ {1} & {0} & {0} & {2} \end{array}\right) \\
				C'=&\left(\begin{array}{rrr|r}  {1} & {0} & {0} & {2} \\ {0} & {1} & {0} & {2} \\ {0} & {0} & {1} & {1} \end{array}\right).
			\end{align*}
		Vi er nu i tilfælde $(3)$, ligningssystemet har da præcis løsningen $x_1=2$, $x_2=2$, $x_3=1$.

		\subsection{1.5}

		Vi bruger her \cite[Sætning 1.2.18]{hesselholt2017}. Totalmatricen:
			\begin{align*}
				A=&\left(\begin{array}{rrr|r} {1} & {1} & {2} & {3} \\ {2} & {-1} & {4} & {0} \\ {1} & {3} & {-2} & {3} \\ {-3} & {-2} & {1} & {0} \end{array}\right)\\
				&\left(\begin{array}{rrr|r} {1} & {1} & {2} & {3} \\ {2} & {-1} & {4} & {0} \\ {1} & {3} & {-2} & {3} \\ {-3} & {-2} & {1} & {0} \end{array}\right)\begin{array}{c} \, \\ -2R_1 \\ -R_1 \\ 3R_3 \end{array}\\
				&\left(\begin{array}{rrr|r} {1} & {1} & {2} & {3} \\ {0} & {-3} & {0} & {-6} \\ {0} & {2} & {-4} & {0} \\ {0} & {1} & {7} & {9} \end{array}\right)\begin{array}{c} -R_4 \\ 3R_4 \\ -2R_4 \\ \, \end{array}\\
				&\left(\begin{array}{rrr|r} {1} & {0} & {-5} & {-6} \\ {0} & {0} & {21} & {21} \\ {0} & {0} & {-18} & {-18} \\ {0} & {1} & {7} & {9} \end{array}\right)\begin{array}{c} -R_4 \\ 3R_4 \\ -2R_4 \\ \, \end{array}\\
				A'=&\left(\begin{array}{rrr|r} {1} & {0} & {0} & {-1} \\ {0} & {1} & {0} & {2} \\ {0} & {0} & {1} & {1} \\ {0} & {0} & {0} & {0}\end{array}\right),
			\end{align*} 
		hvorfra det ses $x=-1$, $y=2$, $z=1$ er den eneste løsning.

		\subsection{1.6}

			Intet nyt, vi bruger \cite[Sætning 1.2.18]{hesselholt2017}.
			\begin{align*}
				A=&\left(\begin{array}{rrrrr|r} {2} & {4} & {-1} & {-2} & {2} & {6} \\ {1} & {3} & {2} & {-7} & {3} & {9} \\ {5} & {8} & {-7} & {6} & {1} & {4} \end{array}\right)\\
				A'=&\left(\begin {array}{rrrrr|r} 1&0&0&0&-3&2\\ 0&1&0&
					-1&2&1\\ 0&0&1&-2&0&2\end {array} \right).
			\end{align*} 
		Vi får da løsningsmængden til
			\begin{align*}
				x =\left(\begin{array}{c} 2+3t \\ 1+s-2t \\ 2+2s \\ s \\ t \end{array}\right).
			\end{align*}

		\subsection{1.7}

		Vi bruger \cite[Sætning 1.2.18]{hesselholt2017}. Den kompleks konjugerede er ofte brugbar her.
			\begin{align*}
				A=&\left(\begin{array}{rr|r} {i} & {2} & {1} \\ {1+2i} & {2+2i} & {3i}  \end{array}\right)\begin{array}{c} -i \\ iR_1 \end{array}\\
				&\left(\begin{array}{rr|r} {1} & {-2i} & {-i} \\ {2i} & {2+4i} & {4i}  \end{array}\right)\begin{array}{c} \, \\ -2i \end{array}\\
				&\left(\begin{array}{rr|r} {1} & {-2i} & {-i} \\ {4} & {8-4i} & {8}  \end{array}\right)\begin{array}{c} \, \\ -4R_1 \end{array}\\
				&\left(\begin{array}{rr|r} {1} & {-2i} & {-i} \\ {0} & {8+4i} & {8+4i}  \end{array}\right)\begin{array}{c} \, \\ 1/80(8-4i) \end{array}\\
				&\left(\begin{array}{rr|r} {1} & {-2i} & {-i} \\ {0} & {1} & {1}  \end{array}\right)\begin{array}{c} 2iR_2 \\ \, \end{array}\\
				A'=&\left(\begin{array}{rr|r} {1} & {0} & {i} \\ {0} & {1} & {1}  \end{array}\right).
			\end{align*} 
		Vi har præcis løsningen $x_1=i$ og $x_2=1$.

		\subsection{1.9}

		Vi bruger \cite[Sætning 1.2.18]{hesselholt2017}.
			\begin{align*}
				A=&\left( \begin {array}{cccc} 1-i&i&3&0\\0&2\,i&2&0 \\ 2&1-i&1+i&0\end {array} \right)\\
				A'=&\left( \begin {array}{cccc} 1 & 0 & 1+i &0 \\0&1& -i &0 \\ 0&0&0&0\end {array} \right)
			\end{align*} 	
		Og løsningen er da $x=0$.

		\subsection{1.10}

		Vi bruger \cite[Sætning 1.2.18]{hesselholt2017}.
			\begin{align*}
				A=&\left(\begin{array}{rrr|r} {1} & {1} & {-1} & {2} \\ {2} & {1} & {1} & {a} \\ {1} & {0} & {2} & {3} \end{array}\right)\begin{array}{c} -R_3 \\ -2R_3 \\ \, \end{array}\\
				&\left(\begin{array}{rrr|r} {0} & {1} & {-3} & {-1} \\ {0} & {1} & {-3} & {a-6} \\ {1} & {0} & {2} & {3} \end{array}\right)\begin{array}{c} \, \\ -2R_1 \\ \, \end{array}\\
				&\left(\begin{array}{rrr|r} {1} & {0} & {2} & {3} \\ {0} & {1} & {-3} & {-1} \\ {0} & {0} & {0} & {a-5} \end{array}\right)
			\end{align*}
		og det ses herfra at for $a\neq 5$ eksisterer der ingen løsninger. Hvis $a=5$ er løsningsmængden givet ved
			\begin{align*}
			  	x =\left(\begin{array}{c} 3-2t \\ -1+3t \\ t \end{array}\right).
			\end{align*}   

		\subsection{M1}

			\paragraph{a} Et homogent ligningssystem tillader altid løsningen $x=0$.

			\paragraph{b} Et ikke homogent ligningssystem kan ikke have 0 som løsning.

		\subsection{M2}

			\paragraph{a} Ja, $x_1+x_2+x_3+x_4+x_5+x_6=\{0,1,2,3,4\}$.

			\paragraph{b} Nej, $x_1+x_2+x_3+x_4+x_5+x_6=0$ 5 gange har f.eks. løsningen $x=0$.

			\paragraph{c} Nej, hvis vi lavede det på reduceret echelon form ville vi se at der ville være en fri variabel altid. Se f.eks. opgave 1.5.

			\paragraph{d} Ja, massere af eksempler.

			\paragraph{e} a) Ja, oftest. b) Nej, hvis ligningerne f.eks. er ens. c) Ja, den ene ligning kan være overflødig og vi har så egentlig en 5 ligninger med 5 ubekendte. d) Med samme argument som før, ligninger kan være overflødige. 

	\section{Opgaver til fordybelse}

		\subsection{1.12} 

			Lad f.eks. $x_2$ til $x_6$ være frie og lad $x_1$ være bestemt af disse.

		\subsection{1.13}

			Hvis det prøves at få totalmatricen på reduceret echelon form fåes
				\begin{align*}
					\left(\begin{array}{cc|c} {1} & {b} & {0} \\ {0} & {ad-bc} & {0} \end{array}\right).
				\end{align*} 
			Det ses herfra at $0$ er den unikke løsning til ligningssystemet hvis og kun hvis $ad-bc\neq 0$

		\subsection{M3}

			Det ses at $d=-1$ og $c=5$ fra de to første betingelser. De to næste betingelse kan skrives op som to ligninger med to ubekendte, hvor det let udregnes at $b=1$ og $a=-2$. Samlet bliver polynomiet $-2x^3+x^2+5x-1$. Kan også opskrive som en matrice og løse systemet derfra.


















%	% !TEX root = ‎⁨‎⁨/../../master.tex

\chapter{Uge 2}

	\section{Basisopgaver}

		\subsection{i}

			\paragraph{1} Da der kun er indgange i diagonalen er den inverse matrice da
				\begin{align*}
					A^{-1}=\left(\begin{array}{lll}{\frac{1}{2}} & {0} & {0} \\ {0} & {\frac{1}{3}} & {0} \\ {0} & {0} & {\frac{1}{4}}\end{array}\right).
				\end{align*} 

			\paragraph{2} $C$ er ikke invertibel da den har determinant $0$.

		\subsection{ii}

			Svaret er 
				\begin{align*}
					\left(\begin{array}{cc}{-5 / 2} & {3 / 2} \\ {2} & {-1}\end{array}\right),
				\end{align*} 
			hvilket enten kan ses ved direkte udregning eller \cite[Eksempel 3.4.3]{hesselholt2017} (Som jeg nok ville mene er et korrollar). 

	\section{Standard opgaver}

		\subsection{0.2} 

			\paragraph{a} Hverken eller, man rammer ikke $-1$, men omvendt bliver $1$ ramt to gange. Billedet er $\real_+$.

			\paragraph{b} Bijektiv. Billedet er $\real$. $g:x\mapsto \sqrt[3]{y}$

			\paragraph{c} Surjektiv, men ej injektiv. Billdet er $\real$.

			\paragraph{d} Den er injektiv og surjektiv. $g:(x_2,y_2)\mapsto \big(\frac{x_1+y_1}{2},\frac{x_1-y_1}{2lv}\big)$

		\subsection{0.6} 

			\paragraph{a} Tag $s,t\in Z$ hvorom der gælder $g(f(s))=g(f(t))$. Siden $g$ er injektiv må $f(s)=f(t)$ og da $f$ er injektiv må $s=t$ og deraf må $g \circ f$ også selv være injektiv.

			\paragraph{b} Tag et vilkårligt $z\in Z$. Da $g$ er surjektiv findes der et $y\in Y$ sådan $g(y)=z$. Da $f$ er surjektiv findes der et $x\in X$ sådan $f(x)=y$. Fra dette må $g \circ f$ også være surjektiv.

			\paragraph{c} Første del følger af $a$ og $b$. Fra \cite[Lemma 0.1.3]{hesselholt2017} ved vi at den inverse eksisterer og den er bijektiv selv. Da $(g \circ f) \circ (f^-1 \circ g^-1)=id$ må $(g \circ f)^-1=(f^-1 \circ g^-1)$ per unikhed af den inverse.

			\paragraph{d} Antag for modstrid at der eksisterer $x_1,x_2 \in X$ sådan at $f(x_1)=f(x_2)$. Så ville vi have at $g(f(x_1))=g(f(x_2))$, men da $f \circ g$ er antaget injektiv kan dette ikke lade sig gøre og $f$ må selv være injektiv

			\paragraph{e} Da $f(X)\subset Y$ må $g(f(X)) \subset g(Y)$. Da $g \circ f$ er antaget surjektiv må $g(f(X))=Z$, hvilket giver at $g(Y)=Z$ og $g$ må selv være surjektiv.

		\subsection{2.16}

			Ved direkte udregning ses det at $A_1=A_3^{-1}$, $A_5=A_6^{-1}$ og $A_7=A_8^{-1}$. 

		\subsection{2.17} 

			Per \cite[sætning 2.4.9]{hesselholt2017} kan vi vise afbildningen er bijektiv ved at vise at $Ax=b$ har præcis en løsning for hvert $b\in \real^3$.  Vi får matricen på reduceret echelon form:
				\begin{align*}
					A|b&=\left(\begin{array}{ccc|c}{0} & {0} & {1 / 4} & {b_1} \\ {0} & {-2} & {0} & {b_2} \\ {3} & {0} & {0} & {b_3} \end{array}\right) \\
					A|b&=\left(\begin{array}{ccc|c}{0} & {0} & {1} & {4b_1} \\ {0} & {1} & {0} & {-1/2b_2} \\ {1} & {0} & {0} & {1/3b_3} \end{array}\right) 
				\end{align*} 
			og det ses for ethvert $b$ har afbildningen præcis løsning $x(4b_1,-1/2b_2,1/3b_3)$.

			Per sætning \cite[2.4.12]{hesselholt2017} kan den inverse matrix findes til at være 
				\begin{align*}
					A^{-1}=\left(\begin{array}{ccc}{0} & {0} & {1/3} \\ {0} & {-1/2} & {0} \\ {4} & {0} & {0} \end{array}\right),
				\end{align*} 
			og $g(y)=A^{-1}y$.

		\subsection{2.20} Lad $f$ være givet ved $f(x_1,x_2,x_3)=(1,x_1,x_2,x_3)$. Denne afbildning er lineæer og injektiv. Denne kunne også være givet ved $f(x_1,x_2,x_3)=(x_1,1,x_2,x_3)$, den er da ikke entydig.

	\section{Opgaver til fordybelse}

		\subsection{0.5} 

			\subsubsection{a} 

			\paragraph{f} Injektivitet: Antag $f(x,y,z)=f(x',y',z')$, så $(x+y,y+z,x+z)=(x'+y',y'+z',x'+z')$. Dette ville give at $x=x'+y'-y$ og så videre at $(x'+y'-y)+z=x'+z'$ og deraf $y'-y=0$ sådan at $y'=y$. På samme måde ville det kunne vises at $x=x'$, $z=z'$ og $f$ er deraf injektiv. Surjektiv: Lad $(a,b,c)$ være en vektor i $\real^3$. Man har da 3 ligninger med 3 ubekendte løses disse fåes
				\begin{align*}
					x=\frac{a-b+c}{2}\\
					y=\frac{a+b-c}{2}\\
					z=\frac{-a+b+c}{2}
				\end{align*} 

			\paragraph{g} $g$ er ej surjektiv. Laves en lignende isolering fåes at $a=b+c$ hvis der skal være en løsning. Punktet $(1,0,0)$ kan derfor ikke rammes.

			\subsubsection{b}

			Den inverse er 
				\begin{align*}
					g\left(\begin{array}{l}{a} \\ {b} \\ {c}\end{array}\right)
					=\left(\begin{array}{l}{\frac{a-b+c}{2}} \\ {\frac{a+b-c}{2}} \\ {\frac{-a+b+c}{2}}\end{array}\right).
				\end{align*} 


		\subsection{0.7} 

			F.eks. ville
				\begin{align*}
					f(x)= \begin{cases}
					\frac{1}{n+1}, \quad &\text{hvis $x=\frac{1}{n}$ hvor $n\in \n$}\\
					x, \quad &\text{ellers}
					\end{cases}
				\end{align*} 
			være en bijektion fra $[0,1] \mapsto [0,1)$

		\subsection{2.18}

			Følger af en udvidelse af \cite[Eksempel 0.1.4]{hesselholt2017}. $a,b,c\neq 0$ og den inverse er da 
				\begin{align*}
					\left(\begin{array}{ccc} {0} & {0} & {1/c} \\ {0} & {1/b} & {0} \\ {1/a} & {0} & {0} \end{array}\right).
				\end{align*} 

		\subsection{2.19}

			Eftervises let ved Maple.

%	% !TEX root = ‎⁨‎⁨/../../master.tex

\chapter{Uge 3}

	\section{Basis opgaver}

		\subsection{i}

			Per \cite[Eksempel 3.2.25]{hesselholt2017} fåes determinanten til $2\cdot 4 - 3 \cdot 1=5$.

		\subsection{ii}

			Vi udregner determinanten af den første matrice til $1 \cdot 3 - 4\cdot 2=-5 \neq 5$. De er da ikke lig med hinanden.

	\section{Standard opgaver}

		\subsection{3.1}

			\paragraph{i} $2\cdot 1 - (-1)\cdot 1=3$. 

			\paragraph{ii} Vi bruger Laplace udvikling langs 3. søjle. (husk fortegn)
				\begin{align*}
					\det \left(\begin{array}{rrr}{2} & {-2} & {3} \\ {4} & {3} & {1} \\ {2} & {0} & {1}\end{array}\right)= 2 \det \left(\begin{array}{rr} -2 & 3 \\ 3 & 1 \end{array}\right) + 1 \det \left(\begin{array}{rr} 2 & -2 \\ 4 & 3 \end{array}\right) = 2(-2-9)+(6+8) = -22+14=-8.
				\end{align*}

			\paragraph{iii} Laplace udvikling langs første søjle
				\begin{align*}
					\det \left(\begin{array}{rrr}{1} & {2} & {1} \\ {5} & {\pi} & {5} \\ {2} & {1 / 2} & {2}\end{array}\right) &= 1 \det \left(\begin{array}{rr} \pi & 5 \\ 1/2 & 2 \end{array}\right) -2 \det \left(\begin{array}{rr} 5 & 5 \\ 2 & 2 \end{array}\right) + 1 \det \left(\begin{array}{rr} 5 & \pi \\ 2 & 1/2 \end{array}\right)\\
					&= 2\pi-5/2-2(10-10)+5/2-2\pi=0
				\end{align*}

		\subsection{3.2}

			\paragraph{A} Vi laver matricen om til en øvre triangulær matrice og bruger \cite[Sætning 3.3.3]{hesselholt2017}. Vi får
				\begin{align*}
					\det \left(\begin{array}{llll}{1} & {2} & {3} & {4} \\ {0} & {0} & {0} & {0} \\ {0} & {0} & {0} & {0} \\ {0} & {0} & {0} & {0}\end{array}\right) =0
				\end{align*}

			\paragraph{B} Vi får determinanten $\cos \theta ^2 + \sin \theta ^2 =1$, hvor den sidste lighed kommer af grundrelation mellem cosinus og sinu.

			\paragraph{C} Det ses at determinanten er $(1+i)(1-i)-2=0$

		\subsection{3.3}

			\paragraph{i} Det ses let ved en triangulation at determinanten er $-24$.

			\paragraph{ii} Det ses let ved en triangulation at determinanten er $-24$.

		\subsection{3.4}

			\paragraph{a} Ja, kan indses ved direkte udregning eller Laplace udvikling af første søjle

			\paragraph{b} Den er ikke på øvre eller nedre triagulær form, men ved at få den på dette ses det at determinanten er $-abc$.

			\paragraph{c} Det er en triagulær matrix, det er sandt.

			\paragraph{d} Med samme argument som i \textbf{b} er det sandt. 

		\subsection{3.5}

			Ved en længere Laplace udvikling fåes determinanten til $a^4b-2ba^2+b$.

		\subsection{3.6}

			\paragraph{i} Determinanten er $4-1=3$.

			\paragraph{ii} Determinanten er $4$.   

			\paragraph{iii} Determinanten er $5$.   

	\section{Øvelser til fordybelse}

		\subsection{3.7}

			Determinanten er $1$. Den drejer rummet, effektivt laver den om på akserne.

		\subsection{3.8}

			\paragraph{i} Determinanten er $(2+i)(1-i)-12i=3-i-12i=3-13i$.

			\paragraph{ii} Determinanten er $1+9i$.

		\subsection{3.9}

			\subsubsection{a}

			Dette bliver exchange matricen (enhedsmatricen spejlet).

			\subsubsection{b}

			Det lader til determinanten skifter fra $-1$ til $1$ for hver anden dimension du går op.

			\subsubsection{c}

			Kan indses ved at 'ombytte' rækker og huske på at fortegnet skifter ved hver operation. Med dette ses det at determinanten er bestem ved
				\begin{align*}
					\bigg\{\begin{array}{ll}{(-1)^{\frac{n}{2}}} & {n \text { er lige }} \\ {(-1)^{\frac{n-1}{2}}} & {n \text { er ulige }}\end{array},
				\end{align*}
			hvilket præcis giver det ønskede.

			\subsubsection{d}

			Når $\frac{n(n-1)}{2}$, hvor $n$ er dimensionen af matricen, er lige.




%	% !TEX root = ‎⁨‎⁨/../../master.tex

\chapter{Uge 4}

	\section{Basis opgaver}

		\subsection{i}

			Vi tjekker om $U$ er stabil med hensyn til vektorrumsstrukturen vha. betingelse 1-3 i \cite[Definition 4.1.4]{hesselholt2017}. 1) $0$ er en del $U$. 2) $(x,0),(y,0)\in U$, da er $(x+y,0)\in U$ oplagt. 3) Antag $(x,0)$, så er $a\cdot (x,0)=(ax,0)\in U$ også .$U$ er også en delmængde af $\real^2$. Der er derfor et underrum.

		\subsection{ii}

			Den opfylder ikke A4 i \cite[Definition 4.1.4]{hesselholt2017} da f.eks. $(-a,-b)$ ikke findes i $V$.

	\section{Standard opgave}

		\subsection{4.4}

			Den opfylder ikke tredje betingelse i \cite[Definition 4.1.4]{hesselholt2017} da $-2\cdot (x_1,x_2,\ldots,x_n)\neq \real^n_{\geq 0}$.

		\subsection{4.5}

			Opfylder ej $V3$. Eks. i $2$-dimensioner: $(1,1)*((1+1i)+(1+2i))=(1,1)*(2+3i)=(\sqrt{13},\sqrt{13})$. Omvendt $(1,1)*(1+i)+(1,1)*(1+2i)=(\sqrt{2},\sqrt{2})+(\sqrt{5},\sqrt{5})=(\sqrt{2}+\sqrt{5},\sqrt{2}+\sqrt{5})$. Følger af trekantsluligheden at det ikke er sandt.

		\subsection{4.7}

			\subsubsection{a}

				Det er oplagt kun $V1-V4$ der kan gå galt. Disse kan let tjekkes at være opfyldt

			\subsubsection{b}

				Brug regneregler og indse at $a=-2$ og $b=3$.

		\subsection{4.8}

			\paragraph{a} Nej, $1/2\cdot (1,1)$ vil ikke være en del af underrummet og 3 vil ej være opfyldt.

			\paragraph{b} Ja, samme argument som i basis opgave $i$.

			\paragraph{c} Nej, 2 ej opfyldt. Tag en vektor hvor førstekoordinatet er $0$ kun og en anden hvor andenkoordinatet $0$ kun. Summen af disse vil ikke ligge i underrummet.

			\paragraph{d} Ja, tjekkes nemt

			\paragraph{e} Nej, $0$ er ikke en del af underrumet.

		\subsection{4.9}

			\subsubsection{a}

				1) Ja, da $0$ er i begge underrum. 2) Hvis $x,y\in V \cap W$, så ligger $x,y\in V$ og $x,y\in W$. Dette betyder at $x+y\in V$ og $x+y\in W$ og videre at $x+y\in V\cap W$. 3) $x\in V \cap W$ så har vi at $a\cdot x\in V$ og $a\cdot x\in W$ og deraf at $a\cdot x\in V \cap W$ pga. de $V$ og $W$ selv er vektorrum.

			\subsubsection{b}

				1) Ja, $0$ er i begge underrum. 2) Hvis $a,b\in V + W$ så har vi $a'+a''=a$ hvor de hver ligger i henholdsvis $V$ og $W$. Samme med $b'+b''=b$. Da $V$ og $W$ hver især er vektorrum følger det at $(a'+b')+(a''+b'')\in V+W$. 3) $a'+a''=a$. $ca=c(a'+a'')=ca'+ca''$ og dette ligger i $V+W$ da de hver især er vektorrum.

		\subsection{M5}

			For at være linæer skal afbildningen opfylde to ting. $f(u+v)=f(u)+f(v)$ og $f(c\cdot u)=cf(u)$. Ergo vi skal tjekke 1) $(cA)^T=cA^T$ og 2) $(A+B)^T=A^T+B^T$. 1) er klart opfyldt og to følger af \cite[Sætning 2.6.7]{hesselholt2017}. Den er dermed linæer.

	\section{Opgaver til fordybelse}

		\subsection{4.6}

			Vi ved allerede matrixsum opfylder A1-A4 i \cite[Definition 4.1.1]{hesselholt2017}. Vi tjekker V1-V4.

%	% !TEX root = ‎⁨‎⁨/../../master.tex

\chapter{Uge 5}

	\section{Basis opgaver}

		\subsection{i}

			$v_1+v_2=(7,3)$. Med hensyn til standardbasen bliver koordinaterne $(7,3)$. Med hensyn til basen $(v_1,v_2)$ bliver det $(1,1)$.

		\subsection{ii}

			Da vi har basen $(v_1,v_2)$ for både domænet og codomænet bliver matricen bare
				\begin{align*}
					A=\left(\begin{array}{rr} {1} & {0} \\ {0} & {3} \end{array}\right).
				\end{align*}

	\section{Standard opgaver}

		\subsection{4.20}

			\paragraph{a} Indses f.eks. vha. \cite[Korollar 4.3.12]{hesselholt2017}, da rangen af matricen basen udspænder er $2$.

			\paragraph{b} Denne bliver 
				\begin{align*}
					P=\left(\begin{array}{rr} {2} & {3} \\ {5} & {7} \end{array}\right).
				\end{align*}

			\paragraph{c} Denne bliver den inverse af $P$ (se f.eks. \cite[Eks. 4.4.13]{hesselholt2017}), dvs.
				\begin{align*}
					P^{-1}=\left(\begin{array}{rr} {-7} & {5} \\ {3} & {-2} \end{array}\right).
				\end{align*}

			\paragraph{d} Per \cite[Eks. 4.4.18]{hesselholt2017} fås koordinaterne 
				\begin{align*}
					y = P^{-1}x = \left(\begin{array}{rr} {-7} & {5} \\ {3} & {-2} \end{array}\right) \begin{pmatrix} x_1 \\ x_2 \end{pmatrix} = \begin{pmatrix} -7x_1 + 5x_2 \\ 3x_1 - 2x_2 \end{pmatrix}.
				\end{align*}

		\subsection{4.21}

			\paragraph{a} Indses f.eks. vha. \cite[Korollar 4.3.12]{hesselholt2017}, da rangen af matricen basen udspænder er $3$.

			\paragraph{b} Denne bliver 
				\begin{align*}
					P=\left(\begin{array}{rrr} {1} & {-1} & {1} \\ {0} & {1} & {-1} \\ {0} & {0} & {1} \end{array}\right).
				\end{align*}

			\paragraph{c} Denne bliver den inverse af $P$ (se f.eks. \cite[Eks. 4.4.13]{hesselholt2017}), dvs.
				\begin{align*}
					P^{-1}=\left(\begin{array}{rrr} {1} & {1} & {0} \\ {0} & {1} & {1} \\ {0} & {0} & {1} \end{array}\right).
				\end{align*}

			\paragraph{d} Per \cite[Eks. 4.4.18]{hesselholt2017} får vi vektoren til 
				\begin{align*}
					y = \left(\begin{array}{rrr} {1} & {1} & {0} \\ {0} & {1} & {1} \\ {0} & {0} & {1} \end{array}\right) \begin{pmatrix} x_1 \\ x_2 \\ x_3 \end{pmatrix} = \begin{pmatrix} x_1 + x_2 \\ x_2 + x_3 \\ x_3 \end{pmatrix}.
				\end{align*}

		\subsection{4.22}

			\paragraph{a} Dette bliver matricen dannet af vektorene $(u_1,u_2,u_3)$, dvs.
				\begin{align*}
					P=\left(\begin{array}{rrr} 1 & 1 & 0 \\ 0 & 2 & 1 \\ 1 & 2 & 1 \end{array}\right).
				\end{align*}

			\paragraph{b} Dette bliver matricen dannet af vektorene $(v_1,v_2)$, dvs.
				\begin{align*}
					P=\left(\begin{array}{rr} 1 & 2 \\ 2 & 3 \end{array}\right).
				\end{align*}

			\paragraph{c} Per \cite[Sætning 4.4.14]{hesselholt2017} er dette $B=Q^{-1}AP$

			\paragraph{d} Dette bliver
				\begin{align*}
					B = Q^{-1}AP = \left(\begin{array}{rrr} {-1} & {16} & {10} \\ {3} & {3} & {0} \end{array}\right).
				\end{align*} 

			\paragraph{e} Tegn.

		\subsection{4.23}

			\paragraph{a} Dette bliver matricen dannet af vektorene $(v_1,v_2,v_3)$, dvs.
			    \begin{align*}
					P=\left(\begin{array}{rrr} 0 & 1 & 1 \\ 1 & 1 & 1 \\ 1 & 2 & 3 \end{array}\right).
				\end{align*}

			\paragraph{b} Per \cite[Sætning 4.4.14]{hesselholt2017} er dette $B=P^{-1}AP$

			\paragraph{c} Dette bliver
				\begin{align*}
					B = P^{-1}AP = \left(\begin{array}{rrr} {3} & {2} & {-1} \\ {0} & {-1} & {1} \\ 0 & 0 & 1 \end{array}\right).
				\end{align*} 

			\paragraph{d} Tegn.

		\subsection{4.24}

			\paragraph{a} Det har rang $3$

			\paragraph{b} Dette bliver matricen dannet af vektorene $(v_1,v_2,v_3)$, dvs.
			    \begin{align*}
					P=\left(\begin{array}{rrr} -1 & 1 & 0 \\ 1 & 0 & 1 \\ 1 & -1 & 1 \end{array}\right).
				\end{align*}

			\paragraph{c} Indses f.eks. fra en tegning at $A=PBP^{-1}$

			\paragraph{d} Vi får
				\begin{align*}
					A = PBP^{-1} = \left(\begin{array}{rrr} {-1} & {1} & {-2} \\ {2} & {2} & {0} \\ 3 & 0 & 2 \end{array}\right).
				\end{align*}
				
			\paragraph{e} Tegn.

		\subsection{4.25}

			\paragraph{a} Dette bliver matricen dannet af vektorene $(v_1, v_2, v_3)$, dvs.
			    \begin{align*}
					P = \left(\begin{array}{rrr} 0 & 1 & 1 \\ 1 & 0 & 1 \\ 1 & 1 & 0 \end{array}\right).
				\end{align*}

			\paragraph{b} Per \cite[Sætning 4.4.14]{hesselholt2017} er dette $B=P^{-1}AP$

			\paragraph{c} Dette bliver
				\begin{align*}
					B = P^{-1}AP = \left(\begin{array}{rrr} {2} & {0} & {0} \\ {0} & {-6} & {0} \\ 0 & 0 & 4 \end{array}\right).
				\end{align*} 

			\paragraph{d} Ja, $f$ er en isomorfi, da $B$ er invertibel (kvadratisk med fuld rang).

			\paragraph{e} Tegn.



	\section{Opgaver til fordybelse}

		\subsection{4.16}

			\paragraph{a} Oplagt.

			\paragraph{b} Skriv det ud og brug at $g$ er lineær.

		\subsection{4.17}
		    Bemærk: Denne opgave påstår fejlagtigt, at $(V, +, \cdot)$ fra opgave 4.7 er et reelt vektorrum. Dette opnås dog først ved at specialisere til $\mathbb{F} = \mathbb{R}$.

			\paragraph{a} Følger af linearitet af matricer

			\paragraph{b} Den naturlige basis er familien $(1, i)$. Således har dette reelle vektorrum dimension $2$.

			\paragraph{c} Den naturlige basis er familien
			\[
			\left(
			\begin{pmatrix} 1 & 0 \\ 0 & 0 \end{pmatrix},
			\begin{pmatrix} 0 & 1 \\ 0 & 0 \end{pmatrix},
			\begin{pmatrix} 0 & 0 \\ 1 & 0 \end{pmatrix},
			\begin{pmatrix} 0 & 0 \\ 0 & 1 \end{pmatrix}
			\right).
			\]
			Således har dette reelle vektorrum dimension $4$.



%	% !TEX root = ‎⁨‎⁨/../../master.tex

\chapter{Uge 6}

	\section{Basis opgaver}

		\subsection{i}

			De er ortogonale da det indre produkt, men ej ortonormale da vektorerne ikke er enhedsvektorer.

		\subsection{ii}

			Det er $\sqrt{4^2+0^2+3^2}=\sqrt{25}=5$.

	\section{Standard opgaver}

		\subsection{6.1}

			\subsubsection{a}

				Vi tjekker \cite[Definition 6.1.1]{hesselholt2017}, oplagt

			\subsubsection{b}

				$|x|=\sqrt{3+4}=\sqrt{7}$. $|y|=\sqrt{3\cdot 16+ 4 \cdot 9}=\sqrt{84}=2\sqrt{21}$. $|z|=\sqrt{3\cdot 3 - 4\cdot 4}=\sqrt{-7}=\sqrt{7}i$.

			\subsubsection{c}

				Indre produktet er $0$, $x$ og $y$ er da ortogonale.

			\subsubsection{d}

				Det er de ikke.

		\subsection{6.2}

			\subsubsection{a}

				Følger af linearitet af integralet.

			\subsubsection{b}

				Det integrerer til $0$ og er derfor ortogonale. $1$ er oplagt en enhedsvektor med hensyn til indre produktet. At $\sqrt{3}(2x-1)$ er følger af en let udregning. 

			\subsubsection{c}

				Normen er $\sqrt{\frac{1}{2n}}$.

		\subsection{6.3}

			Vi har at 
				\begin{align*}
					(x_1+\ldots+x_n)^2=<v,1>^2\leq ||1||^2||v||^2=n(x_1^2+\ldots+x_n^2),
				\end{align*} 
			hvor vi brugte Cauchy-Schwarz i uligheden.

		\subsection{6.4}

			Vi følger \cite[Eksempel 6.1.6]{hesselholt2017}. Indreprodukt af $<x,y>=4$. Vi får da vinklen til $\cos \theta =1$, $\theta =0$.

		\subsection{6.5}

			Vi bruger \cite[Eksempel 6.2.12]{hesselholt2017}. Vi får $(1,-9/5,103/30,-18/30)$ og så normerer vi den.

		\subsection{6.6}

			\subsubsection{a}

				Det ses at de er ortogonale og fra definition af lineært uafhængighed \cite[Definition 4.3.4]{hesselholt2017} er de også det.

			\subsubsection{b}

				Tag f.eks. $w_3=(0,0,1)$. Da er det en basis for $\real^3$ per \cite[Lemma 4.3.9]{hesselholt2017}.

			\subsubsection{c}

				Du ender med standardbasen for $\real^3$.

			\subsubsection{d}

				Oplagt da det er enhedsmatricen.

			\subsubsection{e}

				Oplagt igen.

		\subsection{6.7}

			\subsubsection{a}

				Det en basis for $\real^3$ per \cite[Lemma 4.3.9]{hesselholt2017}, at de er ortogonale eftervises let.

			\subsubsection{b}

				Linearitet følger af at standard indreproduktet er en indreprodukt.

			\subsubsection{c}

				Der regnes og man får da matricen for A til
					\begin{align*}
						\left(\begin{array}{rrrr} 1 & -1 & 1 & 1 \\ 2 & 2 & 0 & 0 \\ 0 & 0 & -1 & 2 \\ -1 & 1 & -1 & 1 \end{array}\right).
					\end{align*} 
				Og ved brug af \cite[Eksempel 4.4.16]{hesselholt2017} bliver B 
					\begin{align*}
						\left(\begin{array}{rrrr} 0 & 0 & 0 & 0 \\ 2 & 1/4 & -1/4 & 1/4 \\ 0 & 7/2 & 1/2 & -1/2 \\ 0 & 1/2 & 7/2 & 1/2 \end{array}\right).
					\end{align*} 

			\subsubsection{d}

				Gøres i Maple\ldots

			\subsubsection{e}

				$f$ har har rang $3$, $f^{\circ 2}$ har rang $2$, $f^{\circ 3}$ har rang $1$, $f^{\circ 4}$ har rang $0$.



	\section{Opgaver til fordybelses}

		\subsection{Opgave 1}

			\subsubsection{a}

				Første del indses let, evt. ved Maple. Dette giver ortogonalitet. At det basis følger af den per definition udspænder $\text{Sig}_3$ og den er lineært uafhængig per \cite[Definition 4.3.4]{hesselholt2017}.

			\subsubsection{b}

				Divider med $\pi$ og vektorerne er ortonormale per a.

			\subsubsection{c}

				Følger af \cite[Sætning 6.2.6]{hesselholt2017} og at basen er ortonormal divideret med $\pi$.


			\subsubsection{d}

				












	




	% !TEX root = ‎⁨‎⁨/../../master.tex

\chapter{Uge 7}

	\section{Basisopgaver}

		\subsection{i}

			Den er lineær, men ikke en isometri. F.eks. bliver $f(1,0)=(2,0)$. $(1,0)$ har norm $1$, mens $(2,0)$ har norm $2$ og det ikke være en isometri per definition

		\subsection{ii}

			Den adjungerede matrix er
				\begin{align*}
						\left(\begin{array}{rrrr} 1 & 1 \\ -1 & 1 \end{array}\right),
				\end{align*} 
			hvilket oplagt ikke er den inverse til matricen A da den har determinant $2$. Den er derfor ikke ortogonal per definition.

	\section{Standardopgaver}

		\subsection{6.8}

			\begin{enumerate}
				\item Alle kvadratiske matricer har egenvektore i $\cx$ (da det er algebraisk lukket), svaret er derfor ja.
				\item Dette er ækvivalent med at endomorfien er normal, hvilket ikke er givet. Svaret er derfor nej.
				\item Ja, per \cite[Korollar 6.2.11]{hesselholt2017}
				\item Ja, per \cite[Korollar 6.2.11]{hesselholt2017}
				\item Nej. Dette gælder hvis og kun hvis $AA^{*}=I$ (Brødtekst s.239)
			\end{enumerate}

		\subsection{6.9}

			\paragraph{a og b} Det ses ved direkte udregning af den inverse til $Q$ at den er normal, da inverse er
				\begin{align*}
					\frac{1}{3}\left(\begin{array}{rrrr} 2 & -1 & 2 \\ 2 & 2 &-1 \\ -1 & 2 & 2 \end{array}\right).
				\end{align*} 

		\subsection{6.10}

			Det ses at matricerne både er unitære og hermitiske per \cite[Definition 6.3.14]{hesselholt2017}. Det ses også ved at hver matrix giver opgaver til polynomiet $\lambda^2-1$, hvilket har løsningerne $\pm 1$.

		\subsection{6.12}

			\paragraph{a} Vi har $1=\det I = \det Q^{*} Q = \det Q^{*} \det Q=(\det Q)^2$.

			\paragraph{b} Samme bevis som i a.

		\subsection{6.15}

			\paragraph{a} Vi får det karakteriske polynomium til $-t^3 + 11t^2 - 38t + 40$. Løses med maple og man får her $5,\,2,\,4$ som løsninger.

			\paragraph{b} Vi udregner først egenvektorene ved at Gauss-eliminere matricen A med rødderne fra det karakteristiske polynomium sat ind. Vi får da vektorene $(-1,-1,1)$, $(1,1,0)$, $(2,-1,1)$. Disse kan derefter laves om til en ortonormal base med Gram Schmidt og det ses også denne udspænder hele $\real^3$. Her f.eks.
				\begin{align*} %From Maple directly 
					[ \left[ \begin {array}{c} -1/3\,\sqrt {3}\\ \noalign{\medskip}-1/3\,
\sqrt {3}\\ \noalign{\medskip}1/3\,\sqrt {3}\end {array} \right] ,
 \left[ \begin {array}{c} 1/6\,\sqrt {6}\\ \noalign{\medskip}1/6\,
\sqrt {6}\\ \noalign{\medskip}1/3\,\sqrt {6}\end {array} \right] ,
 \left[ \begin {array}{c} 1/2\,\sqrt {2}\\ \noalign{\medskip}-1/2\,
\sqrt {2}\\ \noalign{\medskip}0\end {array} \right] ]
				\end{align*} 

			\paragraph{c} Tag vektorene i din ortonormale basis.

		\subsection{6.16}

			\paragraph{a og b} Samme som i 6.15. Egenværdierne bliver $8$, $2$ og $2$. Og en ortonormal basis:
				\begin{align*} %From Maple directly 
					[ \left[ \begin {array}{c} -1/6\,\sqrt {6}\\ \noalign{\medskip}1/6\,
\sqrt {6}\\ \noalign{\medskip}1/3\,\sqrt {6}\end {array} \right] ,
 \left[ \begin {array}{c} 1/2\,\sqrt {2}\\ \noalign{\medskip}1/2\,
\sqrt {2}\\ \noalign{\medskip}0\end {array} \right] , \left[ 
\begin {array}{c} 1/3\,\sqrt {3}\\ \noalign{\medskip}-1/3\,\sqrt {3}
\\ \noalign{\medskip}1/3\,\sqrt {3}\end {array} \right] ]
				\end{align*} 

			\paragraph{c} Tag vektorene i din ortonormale basis.

		\subsubsection{6.17}

			\paragraph{a og b} Samme som i de to foregående. Egenvektore: $1-i, 1+i$. Ortonormal basis:
				\begin{align*} %From Maple directly 
					[ \left[ \begin {array}{c} -1/2\,\sqrt {2}\\ \noalign{\medskip}1/2\,
\sqrt {2}\end {array} \right] , \left[ \begin {array}{c} 1/2\,\sqrt {2
}\\ \noalign{\medskip}1/2\,\sqrt {2}\end {array} \right] ]
				\end{align*} 

			\paragraph{c} Tag vektorene i din ortonormale basis.

	\subsection{Opgaver til fordybelse}

		\subsubsection{6.11}

		\subsubsection{6.13}

		\subsubsection{6.19}

		\subsubsection{6.20}

		\subsubsection{6.21}		

\backmatter

	\bibliography{../master-bib}

\end{document}